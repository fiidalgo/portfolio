\documentclass{report}
\usepackage[utf8]{inputenc}
\usepackage{amsmath, amsfonts, amsthm, graphicx, geometry, lipsum}
\usepackage{hyperref}
\hypersetup{
    colorlinks=true,
    linkcolor=blue,
    urlcolor=red,
    pdftitle={Andrie's Abstract Algebra Notes},
    }
\usepackage{fancyvrb}
\usepackage{fancyhdr, lastpage}
\pagestyle{fancy}
\lhead{Andrie's Abstract Algebra Notes}
% \rhead{Top right}
\rhead{}
\cfoot{Page \thepage\ of \pageref{LastPage}}

% Better spacing for the quantifiers
\let\oldforall\forall
\renewcommand{\forall}{\oldforall \, }
\let\oldexist\exists
\renewcommand{\exists}{\oldexist \: }
\newcommand\existu{\oldexist! \: }
\newcommand{\st}{\quad \mathrm{s.t.} \quad}
\newcommand{\inv}{^{-1}}

\usepackage{etoolbox} %Use carefully!
\patchcmd{\chapter}{\thispagestyle{plain}}{\thispagestyle{fancy}}{}{}

\usepackage[Glenn]{fncychap}

\usepackage{xcolor}
\usepackage{tikz}
\usepackage[most]{tcolorbox}

% Define theorem environments with tcolorbox
\newtcbtheorem{theo}%
  {Theorem}{colback=red!5!white,colframe=red!50!black,separator sign colon}{theorem}

% Define definition environment
\newtcolorbox{definition}[1][]{
  title={Definition},
  colback=blue!5!white,
  colframe=blue!50!black,
  #1
}

% Define example environment
\newtcolorbox{example}[1][]{
  title={Example},
  colback=green!5!white,
  colframe=green!50!black,
  #1
}

% Define note environment
\newtcolorbox{note}[1][]{
  title={Note},
  colback=yellow!5!white,
  colframe=yellow!50!black,
  #1
}

% Define property environment
\newtcolorbox{property}[1][]{
  title={Property},
  colback=purple!5!white,
  colframe=purple!50!black,
  #1
}

% Define methods environment for tests/procedures
\newtcolorbox{methods}[1][]{
  title={Methods},
  colback=orange!5!white,
  colframe=orange!50!black,
  #1
}

\usepackage{siunitx}
\usepackage{setspace}
\onehalfspacing

\usepackage[acronym]{glossaries-extra}
\setabbreviationstyle[acronym]{long-short}
\makeglossaries

\begin{document}

\tableofcontents

\chapter{Groups}
\section{Introduction to Groups}

\begin{definition}[title={Binary Operation}]
Let $G$ be a set. A binary operation on $G$ is a function that 
assigns an ordered pair of $G$ to an element of $G$.

The binary operation \textbf{must} be \textbf{closed}.

Assigns each ordered pair on $G$ to an element of $G$.
\end{definition}

\begin{definition}[title={Group}]
A group is a set with a binary operation that satisfies the following properties:

\begin{enumerate}
  \item \textbf{Associativity}
  \[ (a * b) * c = a * (b * c) \quad \forall a, b, c \in G \]
  \item \textbf{Identity}
  \[ \exists e \st ae = ea = a \quad \forall a \in G \]
  \item \textbf{Inverse}
  \[ \forall a \in G, \exists b \in G \st ab = ba = e\]
\end{enumerate}
\end{definition}

\begin{note}
A group is called \textit{abelian} if it is commutative.
\end{note}

\section{Properties of Groups}

\begin{theo}[separator sign colon]{Unique Identity}{unique-identity}
In a group $G$, there exists \textbf{one unique} identity element.
\end{theo}

\begin{proof}
Suppose $e$ and $e'$ are identities of $G$. Then
\begin{enumerate}
  \item $ae = a, \quad \forall a \in G$
  \item $ae' = a, \quad \forall a \in G$
\end{enumerate}
Using this, observe
\[ ae = ae' \]
\[ a^{-1}ae = a^{-1}ae' \]
\[ e = e' \]
Thus $e = e'$.
\end{proof}

\begin{theo}[colback=red!5!white,colframe=red!50!black]{Cancellation Law}{cancellation-law}
In a group $G$, left and right cancellation law holds.
\end{theo}

\begin{theo}[colback=red!5!white,colframe=red!50!black]{Unique Inverse}{unique-inverse}
\[ \forall a \in G, \existu b \in G \st ab = ba = e\]
\end{theo}

\begin{theo}[colback=red!5!white,colframe=red!50!black]{Shoe-Socks Property}{shoe-socks}
\[ \forall a, b \in G, (ab)\inv = b\inv a\inv \]
\end{theo}

\section{Finite Groups; Subgroups}

\begin{definition}[title={Group and Element Orders}]
\begin{itemize}
  \item \textbf{Order of a Group}
  
  The number of elements in a group is called the \textbf{order}. Denoted as $|G|$.

  \item \textbf{Order of an Element}
  
  The \textbf{order} of an element is the smallest positive integer $n$ such that $a^n = e$.

  If no such integer exists, we say the element has infinite order.

  \item \textbf{Subgroup}
  
  If a subset $H$ of a group $G$ is itself a group under the same operation of $G$,
  we say that $H$ is a \textbf{subgroup} of $G$.
\end{itemize}
\end{definition}

\begin{note}[title={Important Distinction}]
The order of an element should not be confused with the order of a group.
\end{note}

\begin{property}[title={Subgroup Verification}]
There are methods to check if $H$ is a valid subgroup of $G$. You don't need to check that it
satisfies all of the axioms. Some methods to check are below.
\end{property}

\begin{methods}[title={Subgroup Tests}]
\begin{enumerate}
  \item \textbf{One-Step Subgroup Test}
  
  If $ab\inv$ is in $H$ whenever $a$ and $b$ are in $H$, then $H$ is a subgroup of $G$.
  
  \item \textbf{Two-Step Subgroup Test}
  
  If $ab$ is in $H$ whenever $a$ and $b$ are in $H$, and $a\inv$ is in $H$ whenever $a$ is in $H$, then $H$ is a subgroup of $G$.

  \item \textbf{Finite Subgroup Test}
  
  Let $H$ be a nonempty finite subset of a group $G$. If $H$ is closed under the operation of $G$,
  then $H$ is a subgroup of $G$.
\end{enumerate}
\end{methods}

\chapter{Rings}
\section{Introduction to Rings}

\begin{definition}[title={Ring}]
A ring is an algebraic structure consisting of a set equipped with two binary operations that generalize the arithmetic operations of addition and multiplication.
\end{definition}

\chapter{Fields}
\section{Introduction to Fields}

\begin{definition}[title={Field}]
A field is a set on which addition, subtraction, multiplication, and division are defined and behave as the corresponding operations on rational and real numbers do.
\end{definition}

\end{document}